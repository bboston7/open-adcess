\section{Related Work}
\label{sec:related}

Download.com \cite{download-com} provides free downloads of popular programs to
the public.
We borrow their idea of using misleading advertisements to trick users into
downloading programs they may not want in \autoref{sec:fake}.

Rapidshare \cite{rapidshare} was a website that provided rate-limited free
downloads while simultaneously selling a paid download service that was
superior to the free tier.
This technique was meant to entice users to pay for the service.
Unfortunately, as Rapidshare no longer exists it seems that this technique, used alone,
alone is insufficient to cover the costs of running a download service. Or maybe they did not take this technique far enough. He he he.

Wisdom \cite{wisdom} and many other free rags make such heavy use of traditional print ads,
native ads, and classifieds that paper copies may profitably be distributed without charge to health food stores, food cooperatives, and participating Sheetz locations.

We also adopt these techniques to embed in all of our papers in
\autoref{sec:sponsors}.

Edsger W. Dijkstra has previously proposed the revolutionary idea of
capitalizing mathematics \cite{cap-math}.
We expand on this idea with our branded mathematical objects in
\autoref{sec:brands}.
