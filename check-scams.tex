\section{Instant Internet Income}

Digital is the new normal. Digital is the new ubiquity. Digital is the new cuddling. Digital is the new word we use to paper over our alienation with the capitalist mode of production at a time when socialism is possible.

We are now living in a world of 0s and 1s---or, if you prefer, sharks and minnows. The transformative power of this new digital world has been fully realized. Nothing can get more digital. It's either digital or it's not. And by now everything that's ever going to be digital has already become digital. We must accept this fact of this hardscrabble, accept-or-be-accepted world.

With our new digital world comes new digital innovations. Every brand, every person, and every personal brand is now a digital native. Digital is native and native, digital. Every transistor will be able to heal itself without consulting a physical medicine man. Big data, hot data, slow data, medium data, dirty data, long data, cold data, transient data, live data, cubed data, and target-rich data. But data is just a story, a narrative, and though every brand tells an authentic story, a brand is more than a story. The universe is not a story.

Thus, we will need to move beyond digital because digital is just story-telling. What matters is physical. Action potentials sparked by sensory inputs---sight, taste, ESP---and other action potentials. Statues of Jesus being carried by helicopter. Nothing is more authentic for consumers than the physical, than an action potential. In a world where everything is phony, only an action potential---even one involved in the process of telling a brand's story---is real.

Digital embodies the lifestyle and personality of the ACM author. But the ACM author exists in a physical space. Prior to the open-adcess revolution, the ACM author had to transact in this physical space to release a paper (digital locker room talk) in an ACM journal (digital locker). The revolutionary open-adcess model is America-first, digital-second. Now authors need never leave their digital hidey-holes by transferring funds to the ACM and, through fractional-reserve banking, creating physical changes in the non-digital sectors of our digital economy. 

But an author need not remain ensconced in the stately pleasure-dome of digital. In fact, when an author chooses to transact in the physical space, the open-adcess model empowers the author to not only avoid paying the publishing fee but also to get on the road to getting paid for their articles because authors deserve more. Recent accounting innovations surveyed in two review articles and one major motion picture ["Consumer Information: Fake Checks"
https://www.consumer.ftc.gov/articles/0159-fake-checks; "Counterfeit Check Scams: CONSUMER ALERT" http://www.michigan.gov/ag/0,4534,7-164-17337_20942-176356--,00.html; \textit{Bridget Jones's Check Scam} distributed by Universal Pictures] enable the ACM to write large checks to authors. In return, authors simply have to write a small check to cover transaction fees before the large check clears.